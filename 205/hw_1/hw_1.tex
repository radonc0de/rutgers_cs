\documentclass{article}
\title{CS 205 Homework 1}
\author{Keith Lehman, kpl56@scarletmail.rutgers.edu}

\usepackage[margin=0.5in]{geometry}

\begin{document}
\maketitle

\begin{enumerate}
    \item Express propositions as English sentences:
    \begin{enumerate}
        \item You miss the final exam unless you don't have the flu.
        \item You will not pass the course if you have the flu or miss the final.
        \item You have the flu and miss the final exam or you pass the course and attend the final exam.
    \end{enumerate}
    \item State the converse, imverse, and contrapositive of English sentences:
    \begin{enumerate}
        \item
            Converse: If I stay home, it will snow tonight. \\
            Inverse: If it doesn't snow tonight, I won't stay home. \\
            Contrapositive: If I don't stay home, it won't snow tonight. \\
        \item
            Converse: If I go the beach, then it is a sunny summer day. \\
            Inverse: If it isn't a sunny summer day, then I won't go the beach. \\
            Contrapositive: If I don't go to the beach, then it isn't a sunny summer day. \\
        \item
            Converse: When I sleep until noon, I absolutely must have stayed up late the previous night. \\
            Inverse: When I don't stay up late, I don't sleep until noon. \\
            Contrapositive: When I don't sleep until noon, then I must have not stayed up late the previous night. \\
    \end{enumerate}
    \item Prove some statements:
    \begin{enumerate}
        \item $p \iff q$ is equivalent to $(p \land q) \lor (\lnot p \land \lnot q)$: \\  
            \begin{tabular}{ |c|c|c|c|c|c| }
                \hline
                $p$  &  $q$  &  $p \land q$  &  $\lnot p \land \lnot q$  &  $(p \land q) \lor (\lnot p \land \lnot q)$  &  $p \iff q$ \\
                \hline
                T    &  T    &  T            &  F                        &  T                                           &  T          \\
                \hline
                T    &  F    &  F            &  F                        &  F                                           &  F          \\
                \hline
                F    &  T    &  F            &  F                        &  F                                           &  F          \\
                \hline
                F    &  F    &  F            &  T                        &  T                                           &  T          \\
                \hline
            \end{tabular}
        \item $(p \to r) \land (q \to r)$ is equivalent to $(p \lor q) \to r$: \\
            \begin{tabular}{ |c|c|c|c|c| }
                \hline
                $p$ &   $q$ &   $r$ &   $(p \to r) \land (q \to r)$ &   $(p \lor q) \to r$ \\
                \hline
                T   &   T   &   T   &   T   &   T   \\                                                                      
                \hline
                T   &   T   &   F   &   F   &   F   \\
                \hline
                T   &   F   &   T   &   T   &   T   \\
                \hline
                T   &   F   &   F   &   F   &   F   \\
                \hline
                F   &   T   &   T   &   T   &   T   \\
                \hline
                F   &   T   &   F   &   F   &   F   \\
                \hline
                F   &   F   &   T   &   T   &   T   \\
                \hline
                F   &   F   &   F   &   T   &   T   \\
                \hline
            \end{tabular}
    \end{enumerate}
    \item Find an expression equivalent to $p \lor q$ using only $\lnot$ and $\land$: \\
    \begin{enumerate}
        \item Answer: $\lnot (\lnot p \land \lnot q)$. Proof: \\
            \begin{tabular}{ |c|c|c|c| }
                \hline
                $p$ &   $q$ &   $p \lor q$ &   $\lnot (\lnot p \land \lnot q)$ \\
                \hline
                T   &   T   &   T   &   T   \\                                                                      
                \hline
                T   &   F   &   T   &   T   \\
                \hline
                F   &   T   &   T   &   T   \\
                \hline
                F   &   F   &   F   &   F   \\
                \hline
            \end{tabular}
    \end{enumerate}
    \item Prove that $p \lor (\lnot p \land q) \lor (\lnot p \land \lnot q)$ is a tautology.
        \begin{enumerate}
            \item
                \begin{tabular}{ |c|c|c|c| }
                    \hline
                    $p$ &   $q$ &   $p \lor (\lnot p \land q) \lor (\lnot p \land \lnot q)$ \\
                    \hline
                    T   &   T   &   T   \\
                    \hline
                    T   &   F   &   T   \\
                    \hline
                    F   &   T   &   T   \\
                    \hline
                    F   &   F   &   T   \\
                    \hline
                \end{tabular}
        \end{enumerate}
    \item Find a satisfying assignment if one exists for the following, or if not, prove that it's a contradiction: \\ 
        $(p \lor \lnot q) \land (q \lor \lnot r) \land (\lnot r \lor \lnot p) \land (p \lor q \lor \lnot r) \land (\lnot p \lor \lnot q \lor r)$. \\
        Starting from the left, either $p$ is true, $q$ is false, or both. \\
        Test: If $p$ is true, $r$ must be false to make the third epxression true. If $r$ is false, then $q$ must be false in order for the fifth expression to be true. Therefore, $p$ = T, $q$ = F, $r$ = F. \\
    \item What is the negation of the statement “if you take every quiz, you get a cookie”? \\
        Answer: "Despite taking every quiz, you did not receive a cookie". \\
    \item
        \begin{enumerate}
            \item $\exists x\; C(x) \land D(x) \land F(x)$
            \item $\forall x\; C(x) \lor D(x) \lor F(x)$
            \item $\exists x\; C(x) \land -D(x) \land F(x)$
            \item $\forall x\; C(x) \oplus D(x) \oplus F(x)$
            \item $\exists xyz\; C(x) \land D(y) \land F(z)$
        \end{enumerate}
    \item Determine truth values of expressions:
        \begin{enumerate}
            \item True, $x=-1$
            \item True, $x=\frac{1}{2}$
            \item True, works with all real numbers
            \item False, fails with all negative numbers
        \end{enumerate}
    \item
        \begin{enumerate}
            \item $\exists x (\lnot \forall y (P(x) \to Q(y)))$
            \item $\lnot \forall y (P(y) \lor  \lnot \forall x (R(x) \land R(y)))$
        \end{enumerate}
\end{enumerate}
\end{document}
