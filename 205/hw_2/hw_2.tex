\documentclass{article}
\title{CS 205 Homework 2}
\author{Keith Lehman, kpl56@scarletmail.rutgers.edu}

\usepackage[margin=0.5in]{geometry}
\usepackage{/home/keithl/Documents/tex-packages/marvosym/tex/latex/marvosym/marvosym}
\usepackage[dvipsnames]{xcolor}

\begin{document}
\maketitle

\begin{enumerate}
\item Prove or disprove: Every odd number is the difference of two squares. \\
    Proof: Using the numbers $n$ and $n+1$, the difference of both squared would be $(n+1)^2 - n^2 = (n^2+2n+1) - (n^2) = 2n + 1$. The result being $2n+1$ means that the difference between the two squares is 2 times the lower number plus 1. Starting from 0 and 1, this can be used to find 1, the first odd number. Then 1 and 2 would be the 3, then 2 and 3 for 5. Any odd number $m$ can be represented as the difference between $(\frac{m-1}{2} +1)^2$ and $(\frac{m-1}{2})^2$.
\item Prove or disprove: If $x > 3$ and $y < 2$, then $x^2 - 2y > 5$. \\
    Proof: The lowest possible value of $x$ is 4 so the lowest possible value of the first term is $16$ or $(4)^2$. The largest possible value of $y$ of 1, so the largest possible value of the second term is $2$ or $(2)*1$. Under these conditions, the left side is equal to 15, which is greater than 5. As x continues to rise and y continues to decline, this value will continue to rise, but the lowest possible value that can be found is 15.
\item Prove or disprove: For all real numbers $x>0$, there exists a real number $y$ such that $x=y(y+1)$. \\
    Proof: $x=y(y+1)$ can be rerranged as $y^2+y-x$. Since $x$ is a constant, this is in the form of a quadratic equation. The quadratic equation $x^2+x+C$ (the base of the aforementioned one) has at least one real zero as long as $C \leq 0.25$, so for the original equation, since the constant $x$ is on the opposite side, as long as x is greater than -0.25 there will be a real number for $y$ that satisfies the equation. Since the original restrictions are $x > 0$, and this shows that $0 > -0.25$, the original statement is true.  
\item Suppose $2y+3x=3y-4x$, and $x$ and $y$ are not both zero. Prove that $y \neq 0$. \\
    Proof: If $y=0$, the equation would be reduced to $3x=-4x$, which could be true if $x=0$ but both numbers are not allowed to equal zero. \Lightning
\item What is wrong with this "proof"? (see questions) \\
    While the statement does ask to prove that $x \neq 5$ AND $y \neq 12$, these conditions need to also be tested individually to make sure that under all cases, $x$ can never be 5 and $y$ can never be 12. Simply setting $x=5$ and $y=9$ will disprove the theory, as $x+y=14$ even though $x=5$, one of the terms $x$ was not allowed to be.  
\item What is wrong with this "proof"? (see questions) \\
    This proof only tests one specific case. When PROVING by contradiction, it must be shown why the contradiction can NEVER be true, not just one case. A wiser approach for this would be to show that squaring any real number always results in a positive value, which can never be less than 0. 
\item Prove or disprove: If $x$ is even and $y$ is odd, then $y^2-x^2=y+x$. \\
    Disproof: When $x=2$ and $y=1$, then $y^2-x^2=1-4=-3$ while $y+x=2+1=3$. \Lightning
\item Let x and y be any real numbers.
    \begin{enumerate}
        \item Show that $|x| \leq y$ if and only if $-y \leq x \leq y$. \\
            Proof: For positive values of $x$, $|x| \leq y$ can be represented as $x \leq y$ and for negative values $-x \leq y$. For the negative $x$ equation, when both sides are divided by -1 the result is $x \geq -y$. This can be combined with the positive equation to arrive at $-y \leq x \leq y$. Therefore, the original statement is proven. 
        \item Show that $-|x| \leq x \leq |x|$.\\
            Proof: $-|x| \leq x \leq |x|$ when x is a postive number is equal to $-x \leq x \leq x$, which holds for positive values of $x$. When $x$ is a negative number, the original equation is equal to $-(-x) \leq x \leq -x = x \leq x \leq -x$ which holds for negative values of $x$. 
        \item Show that $|x+y| \leq |x| + |y|$ (the "triangle inequality"). \\
            Proof: Scenarios: \\
            ($x$ and $y$ are both positive) $\rightarrow x + y \leq x + y$. Holds. \\
            ($x$ and $y$ are both negative) $\rightarrow -(x+y) \leq -x + -y = -x-y \leq -x-y$. Holds. \\
            ($x$ is positive, $y$ is negative, $|x| > |y|$) $\rightarrow x+y \leq x-y =  y \leq -y$. Since $y$ is negative this Holds. \\
            ($x$ is positive, $y$ is negative, $|x| < |y|$) $\rightarrow -(x+y) \leq x-y = -x-y \leq x-y = -x \leq x$. Since $x$ is positive, this Holds. \\
            ($x$ is postive, $y$ is negative, $|x| = |y|$) $\rightarrow 0 \leq x - y$  Since $x$ is positive and $y$ is negative, this holds. \\
            The last 3 statements can be repeated for when $y$ is positive and $x$ is negative. Therefore, the original statement has been exhaustively proven to be true. \Squaresteel
    \end{enumerate}
\item Prove that for every integer $n$, $n^3$ is even if and only if $n$ is even. \\
    Proof: Assume that an even integer can be presented as 2k, where k is another integer. An odd integer could be represented as 2k+1. If $n$ is an even integer that can be expressed as $2k$, $n^3 = (2k)^3 = 8k^3 = 2(4k^3)$. Therefore, the cube of an even integer will always be even. If $m$ is an odd integer, $m^3$ would be equal to $(2k+1)^3 = 8k^3 + 12k^2 + 6k + 1$ Since $8k^3 + 12k^2 + 6k = 2(4k^3+6k^2+3k)$ is an integer that can be represented as 2(k), the $+1$ at the would mean that this an odd integer. Therefore, the cube of an odd integer will always be an odd integer.
\item Prove or disprove: If we color each of the integers from 1 to 8 two colors (say, red and blue), then for any coloring, there are three integers $x<y<z$ that have the same color, and $y-x=z-y$ (i.e., $x,y,z$ form an arithmetic progression).
    Disproof: The combination [\color{Blue} Blue, Blue, \color{Red} Red, Red, \color{Blue} Blue, Blue, \color{Red} Red, Red] \color{Black} does not have three integers with the same color that would satisfy these conditions. The possiblities are [\color{Blue} (1, 2, 5), (1, 2, 6), (2, 5 , 6), (1, 5, 6), \color{Red} (3, 4, 7), (3, 4, 8), (4, 7, 8), (3, 7, 8)\color{Black}]. None of these conditions satisfy the second part of the statement. 
\end{enumerate}

\end{document}

