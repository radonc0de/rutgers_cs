\documentclass{article}
\title{CS 205 Homework 2}
\author{Keith Lehman, kpl56@scarletmail.rutgers.edu}

\usepackage[margin=0.5in]{geometry}
\usepackage{/home/keithl/Documents/tex-packages/marvosym/tex/latex/marvosym/marvosym}

\begin{document}
\maketitle

\begin{enumerate}
\item Prove or disprove: Every odd number is the difference of two squares. \\
    Disproof: Using 1, the lowest squares are $1^2=1$ and $2^2=4$ which already have a difference of 3. Each successive square beyond these has a large and larger difference in between. Therefore, there does not exist two squares whose difference is equal to 1.
\item Prove or disprove: If $x > 3$ and $y < 2$, then $x^2 - 2y > 5$. \\
    Proof: The lowest possible value of $x$ is 4 so the lowest possible value of the first term is $16$ or $(4)^2$. The largest possible value of $y$ of 1, so the largest possible value of the second term is $2$ or $(2)*1$. Under these conditions, the left side is equal to 15, which is greater than 5. As x continues to rise and y continues to decline, this value will continue to rise, but the lowest possible value that can be found is 15.
\item Prove or disprove: For all real numbers $x>0$, there exists a real number $y$ such that $x=y(y+1)$. \\
    Proof: $x=y(y+1)$ can be rerranged as $y^2+y-x$. Since x is a constant, this is in the form of a quadratic equation. The quadratic equation $x^2+x+C$ (the base of the aforementioned one) has at least one real zero as long as $C \leq 0.25$, so for the original equation, since the constant $x$ is on the opposite side, as long as x is greater than -0.25 there will be a real number for $y$ that satisfies the equation. Since the original restrictions are $x > 0$, and this shows that $0 > -0.25$, the original statement is true.  
\item Suppose $2y+3x=3y-4x$, and $x$ and $y$ are not both zero. Prove that $y \neq 0$. \\
    Disproof: If $y=0$, the equation would be reduced to $3x=-4x$, which could be true if $x=0$ but both numbers are not allowed to equal zero. \Lightning
\item What is wrong with this "proof"? \\
    While the statement does ask to prove that $x \neq 5$ AND $y \neq 12$, these conditions need to also be tested individually to make sure that under all cases, $x$ can never be 5 and $y$ can never be 12. Simply setting $x=5$ and $y=9$ will disprove the theory, as $x+y=14$ even though $x=5$, one of the terms $x$ was not allowed to be.  
\item What is wrong with this "proof"? \\
    This proof only tests one specific case. When PROVING by contradiction, it must be shown why the contradiction can NEVER be true, not just one case. A wiser approach for this would be to show that squaring any real number always results in a positive value, which can never be less than 0. 
\item Prove or disprove: If $x$ is even and $y$ is odd, then $y^2-x^2=y+x$. \\
    Disproof: When $x=2$ and $y=1$, then $y^2-x^2=1-4=-3$ while $y+x=2+1=3$. \Lightning
\item Let x and y be any real numbers.
    \begin{enumerate}
        \item Show that $|x| \leq y$ if and only if $-y \leq x \leq y$.
        \item Show that $-|x| \leq x \leq |x|$.
        \item Show that $|x+y| \leq |x| + |y|$ (the "triangle inequality").
    \end{enumerate}
\end{enumerate}

\end{document}

